\documentclass[a4paper,10pt,french]{article}

\usepackage[
    pdftex,
    margin=2cm,
    headheight=0.4cm,
    headsep=0.8cm,
    footskip=1.2cm,
    nomarginpar,
]{geometry}

\usepackage{xcolor}
\definecolor{linkcolor}{rgb}{0, 0, 0.6}
\usepackage[
    pdftex,
    unicode=true,
    pdfstartview=FitV,
    colorlinks=true,
    citecolor=linkcolor,
    linkcolor=linkcolor,
    urlcolor=linkcolor,
    hyperindex=true,
]{hyperref}

\usepackage{amsmath} % Est-ce qu’on va vraiment s’en servir ?

\usepackage[utf8]{inputenc}
\usepackage[T1]{fontenc}
\usepackage{lmodern}
\usepackage{babel}

\usepackage{graphicx}
\graphicspath{{figures/}}
\usepackage{subcaption}

\usepackage{titling}
\title{Stage d’observation à l’IRAM}
\author{Maximilien Franco, Mélissa Menu, Bruno Pagani \& Jan Vatant--d’Ollone}
\date{14 au 19 mars 2016}

\hypersetup{
    pdfauthor={\theauthor},
    pdftitle={M2 AAIS – Stage en Observatoire},
    pdfsubject={\thetitle},
    pdfkeywords={M2 AAIS, Observatoire, IRAM, Grenade, 30m}
}

\usepackage{fancyhdr}
\pagestyle{fancy}
\fancyhead[L]{\scriptsize\textsc{\thetitle}}
\fancyhead[R]{\scriptsize\textsc{\theauthor}}
\fancyfoot[C]{\thepage}

\begin{document}

\pagenumbering{roman}

% Pour faciliter la mise en forme des pages de garde, on supprime l’indentation
% automatique en début de paragraphe
\setlength{\parindent}{0pt}

% Pas d’en-tête ni de pied pour la première page
\thispagestyle{empty}

\includegraphics[height=2cm]{logo_insu.jpg} \hfill
\includegraphics[height=2cm]{logo_obspm.jpg} \hfill
\includegraphics[height=2cm]{logo_iap.jpg} \hfill
\includegraphics[height=2cm]{logo_psl.png}

\vspace{0.5cm}

\noindent
\begin{minipage}{.5\textwidth}
    \textsc{Master Astronomie, Astrophysique} \\
    \textsc{et Ingénierie Spatiale} \\
    \textit{M2R – Observatoire de Paris}
\end{minipage}%
\begin{minipage}{.5\textwidth}
    \begin{flushright}
        Stage d’Observation 2015--2016 \\
        Maximilien \textsc{Franco}, Mélissa \textsc{Menu}, \\
        Bruno \textsc{Pagani} \& Jan \textsc{Vatant--d’Ollone}
    \end{flushright}
\end{minipage}


\begin{center}

    \vspace{1.5cm}

    \rule[11pt]{5cm}{0.5pt}

    \textbf{\huge \thetitle}

    \rule{5cm}{0.5pt}

    \vspace{1.5cm}

    \parbox{15cm}{\textbf{Résumé} :
    La compréhension des processus conduisant à la formation d'étoiles est une question majeure de l'astrophysique. L'étude de la cinématique et de la composition des nuages protostellaires peut apporter des éléments déterminants dans l'étude de ces mécanismes. Nous proposons ici une présentation succinte et les premiers resulats de l'observation du nuage L1251B réalisé à l'Institut de RadioAstronomie Millimétrique (IRAM) de Grenade à l'aide de l'antenne de 30 m. Un des objectifs centraux de cette observation a été la réalisation de carte d'émission de 12CO, 13CO et C18O ainsi qu'une caractérisation des espèces chimiques présentes dans ce milieu et la détermination précise des vitesses d'éjection spécifiques pour chaque composant.
    }
    

    \vspace{0.5cm}

    \parbox{15cm}{
        \textbf{Mots-clefs} : \it M2 AAIS – Observatoire – Grenade – IRAM – 30m
    }

    \vspace{0.5cm}

    \parbox{15cm}{
        Stage encadré par :

        \textbf{Anaëlle Maury} \\
        \href{mailto:anaelle.maury@cea.fr}{\tt anaelle.maury@cea.fr} / tél. (+33) 1 69 08 36 61 \\
        CEA/DRF/IRFU/SAp/LFEMI – Bât. 709 – Bureau 131 \\
        Service d’Astrophysique (SAp – UMR7158 Astrophysique, Instrumentation et Modélisation) \\
        Laboratoire de Formation des Etoiles et du Milieu Interstellaire (LFEMI) \\
        \url{http://irfu.cea.fr/Sap/}

        \textit{%
            CEA – Centre d’Études de Saclay \\
            Orme des Merisiers \\
            91191 Gif-sur-Yvette CEDEX
        }
    }

    \vspace{0.5cm}

    \hfill \includegraphics[height=2.5cm]{logo_cea.pdf} \hfill \includegraphics[height=2.5cm]{logo_irfu.png} \hfill \includegraphics[height=2.5cm]{logo_aim.jpg} \hfill \null \\
    \vspace{0.5cm}
    \hfill \includegraphics[height=2.5cm]{logo_iram.png} \hfill \includegraphics[height=2.5cm]{logo_upd.png} \hfill \null \\

\end{center}

\vfill
\hfill \thedate

\newpage

\thispagestyle{empty}

\section*{Remerciements}

\tableofcontents

\newpage

\pagenumbering{arabic}

\setlength{\parindent}{16pt}
\setlength{\parskip}{1ex}


\section{Avant-propos}

Afin de profiter au mieux du temps d’observation qui nous était accordé, nous
avons effectué un certain nombre de séances de préparation en amont. Il
s’agissait de prendre connaissance avec le sujet et les objets, les
observations souhaitées, le télescope et son fonctionnement…

TODO: Du blabla sur les sources, leur intérêt.

TODO: Lister les fréquences d’intérêt sous la forme d’un tableau

Côté télescope, le récepteur EMIR permet de faire cela. Il peut fonctionner
dans 4 bandes de fréquences différentes :

TODO: lister les récepteurs et leurs plages.

Pour chaque récepteur, il est possible de connecter plusieurs backends. La
plupart des backends sont indépendants entre eux (c’est a minima le cas de ceux
utilisés, FTS et VESPA), mais pour un détecteur donné, toutes les combinaisons
de branchements ne sont pas possibles. Il était donc d’autant plus important de
réfléchir à l’avance aux \textit{setups} que nous allions utiliser.

Après vérifications des différentes possibilités, et en tenant des éventuels
difficultés météo, nous avons préparés quatre ensembles de possibilités.

TODO: Insérer les figures avec des commentaires sur les choix.

Sur place, la découverte du mode \textit{parallel} de VESPA grâce à la présence
de Gabriel \textsc{Paubert} en tant qu’\textit{Astronomer on duty} nous a
permis d’optimiser plus encore nos \textit{setups}.

\section{Le voyage}

TODO: Une petite photo ? Ou deux. :p

QUESTION: Est-ce qu’on parle de la visite de l’intérieur du télescope, est-ce
qu’on le met ici ?

\section{Les observations}

TODO: W3OH, pointing, focus… Vérifier avec MIRA ce qu’on faisait. Xephem (ou un
nom du genre, je ne sais plus) pour vérifier la position du télescope.

TODO: Quel temps faisait-il chaque jour, qu’est-ce qu’on a choisi comme setups
par conséquent.

TODO: Expliquer les problèmes qu’on a eu (pas position de référence clean alors
qu’en fait elle le sont toutes a posteriori, erreur de position de référence
dans les observations d’une partie des données)

\section{Les données obtenues}

TODO: Lister les raies observées, leurs caractéristiques et par conséquent la
présence d’outflow. Parler de la raie tellurique.

TODO: Ajouter les figures de Jan.

\section*{Conclusion}
\addcontentsline{toc}{section}{Conclusion}

TODO: C’était cool et intéressant. On ajoute une autre photo.

\end{document}
