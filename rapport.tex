\documentclass[a4paper,10pt,french]{article}

\usepackage[
    pdftex,
    margin=2cm,
    headheight=0.4cm,
    headsep=0.8cm,
    footskip=1.2cm,
    nomarginpar,
]{geometry}

\usepackage{xcolor}
\definecolor{linkcolor}{rgb}{0, 0, 0.6}
\usepackage[
    pdftex,
    unicode=true,
    pdfstartview=FitV,
    colorlinks=true,
    citecolor=linkcolor,
    linkcolor=linkcolor,
    urlcolor=linkcolor,
    hyperindex=true,
]{hyperref}

\usepackage{amsmath} % Est-ce qu’on va vraiment s’en servir ?

\usepackage[utf8]{inputenc}
\usepackage[T1]{fontenc}
\usepackage{lmodern}
\usepackage{babel}

\usepackage{graphicx}
\graphicspath{{figures/}}
\usepackage{subcaption}

\usepackage{titling}
\title{Stage d’observation à l’IRAM}
\author{Maximilien Franco, Mélissa Menu, Bruno Pagani \& Jan Vatant--d’Ollone}
\date{14 au 19 mars 2016}

\hypersetup{
    pdfauthor={\theauthor},
    pdftitle={M2 AAIS – Stage en Observatoire},
    pdfsubject={\thetitle},
    pdfkeywords={M2 AAIS, Observatoire, IRAM, Grenade, 30m}
}

\usepackage{fancyhdr}
\pagestyle{fancy}
\fancyhead[L]{\scriptsize\textsc{\thetitle}}
\fancyhead[R]{\scriptsize\textsc{\theauthor}}
\fancyfoot[C]{\thepage}

\begin{document}

\pagenumbering{roman}

% Pour faciliter la mise en forme des pages de garde, on supprime l’indentation
% automatique en début de paragraphe
\setlength{\parindent}{0pt}

% Pas d’en-tête ni de pied pour la première page
\thispagestyle{empty}

\includegraphics[height=2cm]{logo_insu.jpg} \hfill
\includegraphics[height=2cm]{logo_obspm.jpg} \hfill
\includegraphics[height=2cm]{logo_iap.jpg} \hfill
\includegraphics[height=2cm]{logo_psl.png}

\vspace{0.5cm}

\noindent
\begin{minipage}{.5\textwidth}
    \textsc{Master Astronomie, Astrophysique} \\
    \textsc{et Ingénierie Spatiale} \\
    \textit{M2R – Observatoire de Paris}
\end{minipage}%
\begin{minipage}{.5\textwidth}
    \begin{flushright}
        Stage d’Observation 2015--2016 \\
        Maximilien \textsc{Franco}, Mélissa \textsc{Menu}, \\
        Bruno \textsc{Pagani} \& Jan \textsc{Vatant--d’Ollone}
    \end{flushright}
\end{minipage}


\begin{center}

    \vspace{1.5cm}

    \rule[11pt]{5cm}{0.5pt}

    \textbf{\huge \thetitle}

    \rule{5cm}{0.5pt}

    \vspace{1.5cm}

    \parbox{15cm}{\textbf{Résumé} :
        Un petit abstract ? ;)
    }

    \vspace{0.5cm}

    \parbox{15cm}{
        \textbf{Mots-clefs} : \it M2 AAIS – Observatoire – Grenade – IRAM – 30m
    }

    \vspace{0.5cm}

    \parbox{15cm}{
        Stage encadré par :

        \textbf{Anaëlle Maury} \\
        \href{mailto:anaelle.maury@cea.fr}{\tt anaelle.maury@cea.fr} / tél. (+33) 1 69 08 36 61 \\
        CEA/DRF/IRFU/SAp/LFEMI – Bât. 709 – Bureau 131 \\
        Service d’Astrophysique (SAp – UMR7158 Astrophysique, Instrumentation et Modélisation) \\
        Laboratoire de Formation des Etoiles et du Milieu Interstellaire (LFEMI) \\
        \url{http://irfu.cea.fr/Sap/}

        \textit{%
            CEA – Centre d’Études de Saclay \\
            Orme des Merisiers \\
            91191 Gif-sur-Yvette CEDEX
        }
    }

    \vspace{0.5cm}

    \hfill \includegraphics[height=2.5cm]{logo_cea.pdf} \hfill \includegraphics[height=2.5cm]{logo_irfu.png} \hfill \includegraphics[height=2.5cm]{logo_aim.jpg} \hfill \null \\
    \vspace{0.5cm}
    \hfill \includegraphics[height=2.5cm]{logo_iram.png} \hfill \includegraphics[height=2.5cm]{logo_upd.png} \hfill \null \\

\end{center}

\vfill
\hfill \thedate

\newpage

\thispagestyle{empty}

\section*{Remerciements}

\tableofcontents

\newpage

\pagenumbering{arabic}

\section{Avant-propos}

\section{Le voyage}

\section{Les observations}

\section{Les données obtenues}

\end{document}
