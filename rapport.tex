\documentclass[a4paper,10pt,french]{article}

\usepackage[
    xetex,
    margin=2cm,
    headheight=0.4cm,
    headsep=0.8cm,
    footskip=1.2cm,
    nomarginpar,
]{geometry}

\usepackage{xcolor}
\definecolor{linkcolor}{rgb}{0, 0, 0.6}

\usepackage[
    xetex,
    unicode=true,
    pdfstartview=FitV,
    colorlinks=true,
    citecolor=linkcolor,
    linkcolor=linkcolor,
    urlcolor=linkcolor,
    hyperindex=true,
]{hyperref}

\usepackage{amsmath} % Est-ce qu’on va vraiment s’en servir ?

\usepackage[version=4]{mhchem}

\usepackage{siunitx}
\sisetup{
    number-unit-product = \,,
    inter-unit-product = \ensuremath{{}\cdot{}},
    separate-uncertainty = true,
    multi-part-units = single,
    range-phrase = \text{ -- },
    range-units = single
}

\usepackage{enumitem}
\setitemize{itemsep = 1ex}

\usepackage{cleveref}

\usepackage{fontspec}
\usepackage{babel}

\usepackage{graphicx}
\graphicspath{{figures/}}
\usepackage{subcaption}

\usepackage{titling}
\title{Stage d’observation à l’IRAM}
\author{Maximilien Franco, Mélissa Menu, Bruno Pagani \& Jan Vatant--d’Ollone}
\date{14 au 19 mars 2016}

\hypersetup{
    pdfauthor={\theauthor},
    pdftitle={M2 AAIS – Stage en Observatoire},
    pdfsubject={\thetitle},
    pdfkeywords={M2 AAIS, Observatoire, IRAM, Grenade, 30m}
}

\usepackage{fancyhdr}
\pagestyle{fancy}
\fancyhead[L]{\scriptsize\textsc{\thetitle}}
\fancyhead[R]{\scriptsize\textsc{\theauthor}}
\fancyfoot[C]{\thepage}

\newcommand{\setup}{\textit{setup}}
\newcommand{\troismm}{\SI{3}{\milli\meter}}
\newcommand{\unmm}{\SI{1}{\milli\meter}}
\newcommand{\GHz}{\si{\giga\hertz}}

\begin{document}

\pagenumbering{roman}

% Pour faciliter la mise en forme des pages de garde, on supprime l’indentation
% automatique en début de paragraphe
\setlength{\parindent}{0pt}

% Pas d’en-tête ni de pied pour la première page
\thispagestyle{empty}

\includegraphics[height=2cm]{logo_insu.jpg} \hfill
\includegraphics[height=2cm]{logo_obspm.jpg} \hfill
\includegraphics[height=2cm]{logo_iap.jpg} \hfill
\includegraphics[height=2cm]{logo_psl.png}

\vspace{0.5cm}

\noindent
\begin{minipage}{.5\textwidth}
    \textsc{Master Astronomie, Astrophysique} \\
    \textsc{et Ingénierie Spatiale} \\
    \textit{M2R – Observatoire de Paris}
\end{minipage}%
\begin{minipage}{.5\textwidth}
    \begin{flushright}
        Stage d’Observation 2015--2016 \\
        Maximilien \textsc{Franco}, Mélissa \textsc{Menu}, \\
        Bruno \textsc{Pagani} \& Jan \textsc{Vatant--d’Ollone}
    \end{flushright}
\end{minipage}


\begin{center}

    \vspace{1.5cm}

    \rule[11pt]{5cm}{0.5pt}

    \textbf{\huge \thetitle}

    \rule{5cm}{0.5pt}

    \vspace{1.5cm}

    \parbox{15cm}{\textbf{Résumé} :
        La compréhension des processus conduisant à la formation d’étoiles est
        une question majeure de l’astrophysique. Les observations de la
        cinématique et de la composition des nuages proto-stellaires peuvent
        apporter des éléments déterminants dans l’étude de ces mécanismes. Nous
        proposons ici une présentation succincte et les premiers résultats des
        observations du nuage L1251B réalisées à l’Institut de Radio-Astronomie
        Millimétrique (IRAM) de Grenade à l’aide du télescope de
        \SI{30}{\meter} dans le cadre du stage d’observation du M2 AAIS. Un des
        objectifs centraux de cette étude était la réalisation de cartes
        d’émission de \ce{^12CO}, \ce{^13CO} et \ce{C^18O} ainsi qu’une
        caractérisation des espèces chimiques présentes dans ce milieu et la
        détermination précise des vitesses d’éjection spécifiques pour chaque
        composant.
    }

    \vspace{0.5cm}

    \parbox{15cm}{
        \textbf{Mots-clefs} : \it M2 AAIS – Observatoire – Grenade – IRAM – 30m
    }

    \vspace{0.5cm}

    \parbox{15cm}{
        Stage encadré par :

        \textbf{Anaëlle Maury} \\
        \href{mailto:anaelle.maury@cea.fr}{\tt anaelle.maury@cea.fr} / tél. (+33) 1 69 08 36 61 \\
        CEA/DRF/IRFU/SAp/LFEMI – Bât. 709 – Bureau 131 \\
        Service d’Astrophysique (SAp – UMR7158 Astrophysique, Instrumentation et Modélisation) \\
        Laboratoire de Formation des Etoiles et du Milieu Interstellaire (LFEMI) \\
        \url{http://irfu.cea.fr/Sap/}

        \textit{%
            CEA – Centre d’Études de Saclay \\
            Orme des Merisiers \\
            91191 Gif-sur-Yvette CEDEX
        }
    }

    \vspace{0.5cm}

    \hfill \includegraphics[height=2.5cm]{logo_cea.pdf} \hfill \includegraphics[height=2.5cm]{logo_irfu.png} \hfill \includegraphics[height=2.5cm]{logo_aim.jpg} \hfill \null \\
    \vspace{0.5cm}
    \hfill \includegraphics[height=2.5cm]{logo_iram.png} \hfill \includegraphics[height=2.5cm]{logo_upd.png} \hfill \null \\

\end{center}

\vfill
\hfill \thedate

\newpage

\thispagestyle{empty}

\section*{Remerciements}

Nous remercions tout le personnel de l’IRAM pour son accueil chaleureux, que ce
soit les opérateurs, techniciens en tout genre ou les cuisinières, ainsi que
les différents chercheurs (Joshua \textsc{Greenslade}, Alvaro \textsc{Hacar} et
Gabriel \textsc{Paubert}) présents sur place pour les différents échanges que
nous avons eus avec eux. Nous souhaitons remercier également l’ensemble des
personnes ayant participé à l’organisation de ce stage d’observation, que ce
soit à l’IRAM, l’Observatoire, au CEA ou l’Université Paris–Diderot, et
particulièrement à celles qui ont permis la mise en place de ce partenariat il
y a maintenant quelques années.

Enfin, un grand merci à Anaëlle Maury pour son investissement et son
accompagnement tout au long de ce stage, du voyage et de sa préparation, ainsi
que pour les nombreuses et riches discussions que nous avons eues avec elle.

\tableofcontents

\newpage

\pagenumbering{arabic}

\setlength{\parindent}{16pt}
\setlength{\parskip}{1ex}

\section{Avant-propos}

Le temps de télescope étant précieux, nous nous sommes préparés plusieurs mois
avant le début du stage afin de profiter au mieux du temps d’observation qui
nous était accordé. Nous avons effectué un certain nombre de séances de
préparation en amont. Il s’agissait de prendre connaissance avec le sujet et
les objets, les observations souhaitées, le télescope et son fonctionnement…

Les observations avaient pour objectif de mieux comprendre les phénomènes ayant
lieu dans la zone de formation stellaire L1251B. La cartographie effectuée
durant cette semaine permettra de confronter certaines théories, notamment sur
le rôle des jets dans la cinématique et les turbulences au sein des amas
proto-stellaires.

TODO: Un peu plus de blabla sur les sources, leur intérêt. Inclure des figures
envoyées par Anaelle.

Pour étudier un nuage de formation d’étoiles, il est possible de s’intéresser à
différentes molécules et transitions associées. L’intérêt du domaine
millimétrique réside notamment dans la possibilité de construire relativement
facilement de grands télescopes, résultant en de grandes résolutions
angulaires. De plus, l’atmosphère terrestre est peu gênante à ces longueurs
d’ondes. Seule une grande quantité de vapeur d’eau dans l’atmosphère peut nous
poser problème, ce qui peut être le cas par mauvais temps.

L’une des molécules les plus standards dans l’étude des nuages est \ce{CO},
présente partout et qui permet de tracer les contours à grande échelle des
nuages. D’autre part, son abondance est telle que celle de ses isotopes n’est
pas négligeable non plus, \ce{^13CO} et \ce{C^18O} permettent de tracer un peu
mieux les régions denses. En cas de présence d’\textit{outflows}, ces raies
sont les plus susceptibles de permettre un bon traçage. D’autres molécules
permettent de tracer les régions très denses, chaudes ou froides… Lors de la
préparation, une liste des transitions potentiellement observables fût établie.
Elle est présentée \cref{tab:raies_setup}. Évidemment, un certain nombre
d’entre elles ne seront pas forcément présentes, et d’autres peuvent
apparaître.

\begin{table}[ht]
    \centering
    \begin{tabular}{ccc}
        \hline
        \hline
        Molécule & Transition & Fréquence (\GHz) \\
        \hline
        \ce{H^13CO+}   &(1-0)   & \num{ 86.75429} \\
        \ce{HCN}       &(1-0)   & \num{ 88.63185} \\
        \ce{HCO+}      &(1-0)   & \num{ 89.18852} \\
        \ce{N2H+}      &(1-0)   & \num{ 93.17370} \\
        \ce{CH3OH}     &(2-1)   & \num{ 96.74138} \\
        \ce{CS}        &(2-1)   & \num{ 97.98095} \\
        \ce{C^18O}     &(1-0)   & \num{109.78217} \\
        \ce{^13CO}     &(1-0)   & \num{110.20135} \\
        \ce{^12CO}     &(1-0)   & \num{115.27120} \\
        \ce{e-CH3OH}   &(5-4)   & \num{216.94560} \\
        \ce{SiO}       &(5-4)   & \num{217.10498} \\
        \ce{C^18O}     &(2-1)   & \num{219.56035} \\
        \ce{H2{}^13CO} &(3-2)   & \num{219.90853} \\
        \ce{SO}        &(6-5)   & \num{219.94944} \\
        \ce{^13CO}     &(2-1)   & \num{220.39868} \\
        \ce{CH3CN}     &(12-11) & \num{220.64108} \\
        \ce{^12CO}     &(2-1)   & \num{230.53800} \\
        \ce{^13CS}     &(5-4)   & \num{231.22069} \\
        \ce{N2D+}      &(3-2)   & \num{231.32183} \\
        \hline
    \end{tabular}
    \caption{Liste des fréquences des principales transitions des molécules
    attendues}
    \label{tab:raies_setup}
\end{table}

Ces raies seront plus ou moins intenses et larges en fonction de leurs
caractéristiques intrinsèques et de la densité des espèces correspondantes,
mais pourront également présenter plusieurs composantes avec la présence
d’éventuels \textit{outflows}.

Nous voudrions donc pouvoir disposer d’une assez bonne résolution angulaire
pour réaliser la cartographie de la région d’intérêt, une relativement bonne
résolution spectrale pour bien séparer les raies et leurs éventuelles
composantes, tout en étant assez large bande pour couvrir plusieurs raies à la
fois.

Côté télescope, le récepteur EMIR permet de faire cela. Il peut fonctionner
dans 4 bandes de fréquences différentes, présentées \cref{tab:emir}. Par
rapport aux raies à observer, on peut constater qu’elle tombe dans les
récepteurs E0 et E2. En fait, c’est par construction, car les possibilités de
fonctionnement d’EMIR ont été pensées en fonction des raies connues et d’autres
paramètres techniques, et nos listes de raies ont été elles-mêmes conçues pour
correspondre à ces détecteurs, pouvant fonctionner de pair.

\begin{table}[ht]
    \centering
    \begin{tabular}{cccc}
        \hline
        \hline
        Bande & Fréquence (\GHz)    & LSB (\GHz)          & USB (\GHz)          \\
        E0    & \numrange{ 73}{117} & \numrange{ 73}{97 } & \numrange{ 89}{117} \\
        E1    & \numrange{125}{184} & \numrange{125}{168} & \numrange{141}{184} \\
        E2    & \numrange{202}{274} & \numrange{202}{268} & \numrange{217}{274} \\
        E3    & \numrange{277}{335} & \numrange{277}{335} & \numrange{293}{350} \\
        \hline
    \end{tabular}
    \caption{Caractéristiques d’EMIR. EMIR se règle à une fréquence donnée, et
    dispose 4 blocs autour de cette fréquence, 2 au-dessous et 2 au-dessus.}
    \label{tab:emir}
\end{table}

Pour chaque récepteur, il est possible de connecter plusieurs
\textit{backends}. La plupart des \textit{backends} sont indépendants entre eux
(c’est presque le cas de ceux utilisés, FTS et VESPA), mais pour un détecteur
donné, toutes les combinaisons de branchements ne sont pas possibles. Il était
donc d’autant plus important de réfléchir à l’avance aux \textit{setups} que
nous allions utiliser. FTS propose des détecteurs très larges bandes (environ
\SI{8}{\giga\hertz} chacun, pour un total de 8), mais disposant évidemment
d’une résolution spectrale limitée. Toutes les combinaisons des 8 détecteurs ne
sont pas possibles, et les choix de positionnement ou résolution sont très
limités. VESPA propose au contraire des détecteurs à bande très étroite, mais
disposant d’une grande sensibilité et offrant de très nombreux réglages.

Après vérifications des différentes possibilités, et en tenant compte des éventuelles
difficultés météo, nous avons préparé un ensemble de quatre \textit{setups},
présentés
\cref{fig:specsetup_1mm,fig:specsetup_3mm,fig:specsetup_3mm_1mm_u,fig:specsetup_3mm_1mm_l}.
Le but était d’avoir à minima les raies de \ce{CO} et de ses isotopes, et
d’essayer de maximiser au mieux possible le nombre de raies observées et la
résolution à laquelle celles-ci l’étaient.

\begin{figure}[ht]
    \centering
    \includegraphics[width=0.9\textwidth]{specsetup-1mm.pdf}
    \caption{Plages de fréquences sélectionnées pour les observations à \unmm.}
    \label{fig:specsetup_1mm}
\end{figure}

\begin{figure}[ht]
    \centering
    \includegraphics[width=0.9\textwidth]{specsetup-3mm.pdf}
    \caption{Plages de fréquences sélectionnées pour les observations à \troismm.}
    \label{fig:specsetup_3mm}
\end{figure}

\begin{figure}[ht]
    \centering
    \includegraphics[width=0.9\textwidth]{specsetup-3mm-1mm-u.pdf}
    \caption{Plages de fréquences sélectionnées pour les observations combinées à \unmm{} et \troismm.}
    \label{fig:specsetup_3mm_1mm_u}
\end{figure}

\begin{figure}[ht]
    \centering
    \includegraphics[width=0.9\textwidth]{specsetup-3mm-1mm-l.pdf}
    \caption{Plages de fréquences complémentaires pour les observations combinées à \unmm{} et \troismm.}
    \label{fig:specsetup_3mm_1mm_l}
\end{figure}

Sur place, la découverte du mode \textit{parallel} de VESPA grâce à la présence
de Gabriel \textsc{Paubert} en tant qu’\textit{Astronomer on duty} nous a
permis d’optimiser plus encore nos \textit{setups}.

Si la résolution angulaire est moins bonne à \troismm, la dépendance en la
qualité de la météo est également bien moindre, ce qui permet d’observer quand
même en cas de mauvais temps.

\section{Le voyage}

TODO: Une petite photo ? Ou deux. :p Je vous laisse le choix, personellement j'ai pas ce qui faut pour ça (essentielement des nuages)

Le voyage a été très bien organisé en amont. Les dates choisies pour le voyage
étaient parfaitement adaptées aux créneaux réservés pour nous au télescope (pas
de précipitation). Nous avons été très bien accueillis au télescope, autant par
les chercheurs présents que par le personnel de l’IRAM. Nous avons apprécié
tout particulièrement la présentation qui nous a été donnée le jour de notre
arrivée, ainsi que les échanges, fugaces mais riches, aux heures de repas.

Arrivés à Grenade relativement tôt, nous avons pu profiter de la fin
d’après-midi pour découvrir la ville et admirer l’Alhambra, et même apercevoir
l’antenne du télescope. Le lendemain nous partions tôt pour celui-ci, avec
quelques autres chercheurs et de nombreux gens de l’IRAM. Nous avons partagé un
déjeuner frugal avec tous les passagers à mi-chemin comme le veut la tradition.
Nous poursuivîmes ensuite notre ascension vers la station de ski.

Il est particulièrement agréable, arrivé en bas de la station, de doubler tous
les skieurs et autres surfeurs ainsi que l’ESF locale pour monter dans les
œufs. La partie du trajet effectuée en chenille donne un certain cachet à
l’expédition, même si beaucoup semblent nostalgiques de l’ancien véhicule, qui
donnait par son inconfort une véritable impression d’exploration aventurière.

Nous avons également pu profiter du premier jour sur place, sans observations,
pour nous promener un peu et faire quelques photos. Le lieu est vraiment
magnifique, et les paysages, dénués de touristes passé 17 h bien que le Soleil
soit encore relativement haut dans le ciel, splendides, particulièrement lors
du coucher de Soleil.

\begin{figure}[ht]
    \centering
    \includegraphics[width=0.9\textwidth]{30m.jpg}
    \caption{Le télescope au coucher du Soleil.}
\end{figure}

Nous avons également eu l’occasion de visiter l’intérieur du télescope en fin
de séjour, ce qui nous a permis de mieux appréhender l’instrumentation mise en
place et de réaliser la complexité de cette technologie, toujours fonctionnelle
après plus de 30 ans de loyaux services.

Le dernier jour, nous arrivâmes à Grenade en fin d’après-midi, ce qui nous
permis de faire un point sur le séjour avec Anaëlle avant d’aller visiter
l’Alhambra de nuit. Le retour à Paris se passa sans encombre.

\section{Les observations}

Le premier jour, nous avons commencé par prendre en main les différents outils
présents sur place : PaKo, Xephem, MIRA, TAPAS…

Les tâches étaient réparties sur 3 à 4 postes :
\begin{itemize}
    \item Un poste contrôlait Xephem pour vérifier le positionnement du
          télescope dans le ciel, sa position vis-à-vis du
          pointage demandé et sa position par rapport au Soleil. Il déterminait aussi le
          choix de la topologie « high » ou « low » en fonction de la
          trajectoire des sources sur le ciel et des contraintes du télescope…
    \item Un second poste contrôlait PaKo pour gérer les instructions envoyées
          au télescope et vérifier la bonne prise en compte de celles-ci. C’est
          également à ce poste que le « cahier d’observations » TAPAS était
          rempli.
    \item Un troisième poste servait à analyser les données à l’aide de MIRA
          pour fournir les corrections à apporter dans PaKo et visualiser les
          résultats préliminaires.
    \item Un éventuel quatrième poste servait à traiter les données des jours
          précédents et celles s’ajoutant au fur et à mesure des observations
          pour en tirer les premiers résultats.
\end{itemize}

\begin{figure}[ht]
    \centering
    \includegraphics[width=0.9\textwidth]{control_room.jpg}
    \caption{La salle de contrôle, avec les différents postes.}
\end{figure}

Toujours le premier jour, nous avons commencé dans un premier temps à nous
familiariser avec les différentes étapes à réaliser en début de chaque
observation :
\begin{itemize}
    \item Choix du \setup{} en fonction de la météo et des observations
          souhaitées, puis \textit{tuning} par l’opérateur. Nous remercions à
          cette occasion Manuel \textsc{Ruiz}, dit \textit{Manolo}, pour son incroyable
          efficacité ;
    \item Calibration puis \textit{pointing} sur une source de référence (ici,
          \textit{W3OH}), alternance de \textit{focus} et \textit{pointing} en
          fonction de leurs résultats respectifs ;
    \item Mesure (\textit{track}) sur une source de référence proche de la
          source observée pour obtenir un fond de carte ;
    \item Relevés sur la source proprement dite.
\end{itemize}

Nous observions chaque jour de 8 h 30 à 16 h. Le premier jour, nous avons
récupéré la main tôt : la météo fût catastrophique durant la nuit, et était
assez mitigée le matin. Nous avons donc décidé de commencer sur le \setup{} à
\troismm.

Comme c’était notre première session d’observation sur la source, il nous
fallait trouver une position de référence propre. En effet, l’un des problèmes
majeurs des observations à ces longueurs d’ondes est de trouver des zones
proches de la source sans émission, alors que \ce{^12CO} est une molécule
très abondante, particulièrement dans les nuages environnant les cœurs
pré-stellaires. En préparation, nous avions listé les coordonnées
d’emplacements potentiellement exempts d’émissions. Une fois les premières
étapes systématiques effectuées, nous avons donc effectué des relevés à chacune
des ces positions. À notre grand désespoir, aucune ne s’est avérée propre, et
elles étaient d’ailleurs toutes très similaires, pour une raison évidente
\textit{a posteriori} (\textit{cf.} \cref{sec:resultats}).

Le temps de réaliser tout ceci, la météo s’était fortement dégagée, mais
restait encore limite pour basculer à \unmm. Nous avons donc décidé d’être
partiellement conservateurs et de basculer sur le \setup{} intermédiaire
combinant \unmm{} et \troismm. Ainsi, si la météo se dégradait, nous aurions au
moins les données à \troismm{} à exploiter. En pratique, le temps s’est
encore plus dégagé, pour aboutir à une opacité quasiment nulle. En fin de
journée, nous avons gardé un peu de temps pour persister dans nos recherches
d’une position de référence propre, toujours sans plus de succès.

Le second jour, la météo s’annonçait superbe dès le début : nous sommes partis
directement sur le \setup{} à \unmm. Enfin, presque. Nous avons eu quelques
soucis de fichiers de configuration manquants, heureusement rapidement
restaurés grâce à Gabriel.

Le troisième et dernier jour, la météo n’était pas très bonne. Nous avons
décidé d’utiliser uniquement le \setup{} à \troismm. Cependant, comme c’était
la première fois que nous utilisions ce \setup{} pour observer la source, nous
n’avions pas repéré une erreur présente dans le script \textit{OTF}. En effet,
l’\textit{offset} indiqué pour la position de référence était nul, car nous
devions le régler une fois cette position choisie. Ce que nous avions fait pour
l’autre script, mais pas celui-ci. Les premières données nous paraissaient
évidemment aberrantes, sans que nous ne comprenions pourquoi. Une fois de plus,
c’est l’arrivée de Gabriel, qui revenait du ski, qui nous a débloquée. Il a
tout suite suggéré de vérifier la position de référence, et nous nous sommes
aperçus de notre erreur. Heureusement, celle-ci était corrigible moyennant
quelques \textit{on/off} en \textit{frequency switching} avec la position de
référence et Anaëlle a réussi à réparer toutes les données.

Nous avons conclu cette dernière journée d’observations par une visite de
l’intérieur du télescope, très intéressante et instructive !

\begin{figure}[ht]
    \centering
    \includegraphics[width=0.9\textwidth]{inside.jpg}
    \caption{À l’intérieur du télescope.}
\end{figure}

\section{Les données obtenues}
\label{sec:resultats}

Les résultats obtenus le premier jour ont été combinés avec ceux des jours
précédents, de sorte à obtenir des cartes plus précises. En effet, l’un des
intérêts de nos \textit{setups} était de pouvoir combiner facilement les
données issues des différentes observations.

En plus d’une partie des raies prévues au \cref{tab:raies_setup} (notamment
\ce{CO} et tous ses isotopes), d’autres raies ont été observées, dont certaines
sont listées \cref{tab:raies_bonus}. La présence de \ce{C^17O} indique du gaz
vraiment dense.

D’autre part, les raies de \ce{CO} (et isotopes) présentent deux composantes,
indiquant la présence d’un ou deux \textit{outflows}. Le fait que même
\ce{C^17O} soit concerné montre que ceux-ci sont vraiment très forts. La raie
de \ce{^12CO} présentait également une troisième composante, mais celle-ci a
disparu une fois le spectre de la source de référence soustrait : en effet, la
raie observée sur toutes les sources de référence avait un décalage en vitesse
différent, impliquant une provenance d’un quelconque avant plan. En fin de
compte, elle n’était pas due au nuage de gaz ou au milieu interstellaire en
avant plan, il s’agissait en fait tout simplement de la raie tellurique ! D’où
une présence sur l’ensemble des spectres mesurés. Nos positions de référence
étaient donc bien propres finalement. Merci à Laurent \textsc{Pagani} pour
cette remarque judicieuse.

\begin{table}[ht]
    \centering
    \begin{tabular}{ccc}
        \hline
        \hline
        Molécule & Transition & Fréquence (\GHz) \\
        \hline
        \ce{HC3N}  & (12-11) & \num{109.17363} \\
        \ce{C^17O} & (1-0)   & \num{112.35898} \\
        \ce{C^17O} & (1-0)   & \num{112.36001} \\
        \ce{CN}    & (1-0)   & \num{113.12337} \\
        \ce{CN}    & (1-0)   & \num{113.14416} \\
        \ce{CN}    & (1-0)   & \num{113.17049} \\
        \ce{CN}    & (1-0)   & \num{113.19128} \\
        \ce{CN}    & (1-0)   & \num{113.48812} \\
        \ce{CN}    & (1-0)   & \num{113.49097} \\
        \ce{CN}    & (1-0)   & \num{113.49964} \\
        \ce{CN}    & (1-0)   & \num{113.50891} \\
        \ce{CN}    & (1-0)   & \num{113.52043} \\
        \hline
    \end{tabular}
    \caption{
        Liste des fréquences de certaines transitions supplémentaires
        observées. Il y a deux groupes de raies pour \ce{CN}, et les neufs
        raies ont été observées avec les bons rapports.
    }
    \label{tab:raies_bonus}
\end{table}

\begin{figure}[ht]
    \centering
    \includegraphics[width=0.9\textwidth]{12COfull.png}
    \caption{12COfull}
\end{figure}

\begin{figure}[ht]
    \centering
    \includegraphics[width=0.9\textwidth]{12COblue.png}
    \caption{12COblue}
\end{figure}

\begin{figure}[ht]
    \centering
    \includegraphics[width=0.9\textwidth]{12COred.png}
    \caption{12COred}
\end{figure}

TODO: Commenter les figures de Jan.

\begin{figure}[ht]
    \centering
    \includegraphics[height=6cm]{mapC17O.png}
    \caption{
        Carte des flots observés de CO. En bleu le $^{12}CO$ « blueshifté ». En
        rouge le flot « redshifté ». Les isocontours de vitesse sont pris avec
        comme référence la vitesse de la raie principale de $^{12}CO$ (2→1) à
        \SI{230.538}{\giga\hertz}, observée dans le \setup{} à \unmm. Les bins
        de densité ont été réalisés avec la raie de $^{17}CO$, meilleur traceur
        du gaz dense au repos.
    }
    \label{mapC17O}
\end{figure}

\section*{Conclusion}
\addcontentsline{toc}{section}{Conclusion}

Ce stage fut enrichissant pour chacun d’entre nous. Au-delà de nous apprendre à
manipuler le télescope, il nous a permis de découvrir le fonctionnement d’un
observatoire international très actif. L’objectif initial du projet a été
atteint, puisque les résultats que nous avons obtenus sont tout à fait
exploitables. Nous aimerions retravailler sur ces résultats dans quelques mois,
afin d’en approfondir l’interprétation.

Ce fut une expérience mémorable, et passionante y compris pour des étudiants
qui ne sont a priori pas portés sur l’observation. Nous sommes très heureux
d’avoir pu prendre part à cette expérience, et espérons pour les promotions
suivantes que ce partenariat sera reconduit pour les années à venir.

TODO: Une photo de la fin ?

\end{document}
