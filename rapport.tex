\documentclass[a4paper,10pt,french]{article}

\usepackage[
    pdftex,
    margin=2cm,
    headheight=0.4cm,
    headsep=0.8cm,
    footskip=1.2cm,
    nomarginpar,
]{geometry}

\usepackage{xcolor}
\definecolor{linkcolor}{rgb}{0, 0, 0.6}
\usepackage[
    pdftex,
    unicode=true,
    pdfstartview=FitV,
    colorlinks=true,
    citecolor=linkcolor,
    linkcolor=linkcolor,
    urlcolor=linkcolor,
    hyperindex=true,
]{hyperref}

\usepackage{amsmath} % Est-ce qu’on va vraiment s’en servir ?

\usepackage[utf8]{inputenc}
\usepackage[T1]{fontenc}
\usepackage{lmodern}
\usepackage{babel}

\usepackage{graphicx}
\graphicspath{{figures/}}
\usepackage{subcaption}

\usepackage{titling}
\title{Stage d’observation à l’IRAM}
\author{Maximilien Franco, Mélissa Menu, Bruno Pagani \& Jan Vatant--d’Ollone}
\date{14 au 19 mars 2016}

\hypersetup{
    pdfauthor={\theauthor},
    pdftitle={M2 AAIS – Stage en Observatoire},
    pdfsubject={\thetitle},
    pdfkeywords={M2 AAIS, Observatoire, IRAM, Grenade, 30m}
}

\usepackage{fancyhdr}
\pagestyle{fancy}
\fancyhead[L]{\scriptsize\textsc{\thetitle}}
\fancyhead[R]{\scriptsize\textsc{\theauthor}}
\fancyfoot[C]{\thepage}

\begin{document}

\pagenumbering{roman}

% Pour faciliter la mise en forme des pages de garde, on supprime l’indentation
% automatique en début de paragraphe
\setlength{\parindent}{0pt}

% Pas d’en-tête ni de pied pour la première page
\thispagestyle{empty}

\includegraphics[height=2cm]{logo_insu.jpg} \hfill
\includegraphics[height=2cm]{logo_obspm.jpg} \hfill
\includegraphics[height=2cm]{logo_iap.jpg} \hfill
\includegraphics[height=2cm]{logo_psl.png}

\vspace{0.5cm}

\noindent
\begin{minipage}{.5\textwidth}
    \textsc{Master Astronomie, Astrophysique} \\
    \textsc{et Ingénierie Spatiale} \\
    \textit{M2R – Observatoire de Paris}
\end{minipage}%
\begin{minipage}{.5\textwidth}
    \begin{flushright}
        Stage d’Observation 2015--2016 \\
        Maximilien \textsc{Franco}, Mélissa \textsc{Menu}, \\
        Bruno \textsc{Pagani} \& Jan \textsc{Vatant--d’Ollone}
    \end{flushright}
\end{minipage}


\begin{center}

    \vspace{1.5cm}

    \rule[11pt]{5cm}{0.5pt}

    \textbf{\huge \thetitle}

    \rule{5cm}{0.5pt}

    \vspace{1.5cm}

    \parbox{15cm}{\textbf{Résumé} :
    La compréhension des processus conduisant à la formation d'étoiles est une question majeure de l'astrophysique. L'étude de la cinématique et de la composition des nuages protostellaires peut apporter des éléments déterminants dans l'étude de ces mécanismes. Nous proposons ici une présentation succinte et les premiers resulats de l'observation du nuage L1251B réalisé à l'Institut de RadioAstronomie Millimétrique (IRAM) de Grenade à l'aide de l'antenne de 30 m. Un des objectifs centraux de cette observation a été la réalisation de carte d'émission de 12CO, 13CO et C18O ainsi qu'une caractérisation des espèces chimiques présentes dans ce milieu et la détermination précise des vitesses d'éjection spécifiques pour chaque composant.
    }
    

    \vspace{0.5cm}

    \parbox{15cm}{
        \textbf{Mots-clefs} : \it M2 AAIS – Observatoire – Grenade – IRAM – 30m
    }

    \vspace{0.5cm}

    \parbox{15cm}{
        Stage encadré par :

        \textbf{Anaëlle Maury} \\
        \href{mailto:anaelle.maury@cea.fr}{\tt anaelle.maury@cea.fr} / tél. (+33) 1 69 08 36 61 \\
        CEA/DRF/IRFU/SAp/LFEMI – Bât. 709 – Bureau 131 \\
        Service d’Astrophysique (SAp – UMR7158 Astrophysique, Instrumentation et Modélisation) \\
        Laboratoire de Formation des Etoiles et du Milieu Interstellaire (LFEMI) \\
        \url{http://irfu.cea.fr/Sap/}

        \textit{%
            CEA – Centre d’Études de Saclay \\
            Orme des Merisiers \\
            91191 Gif-sur-Yvette CEDEX
        }
    }

    \vspace{0.5cm}

    \hfill \includegraphics[height=2.5cm]{logo_cea.pdf} \hfill \includegraphics[height=2.5cm]{logo_irfu.png} \hfill \includegraphics[height=2.5cm]{logo_aim.jpg} \hfill \null \\
    \vspace{0.5cm}
    \hfill \includegraphics[height=2.5cm]{logo_iram.png} \hfill \includegraphics[height=2.5cm]{logo_upd.png} \hfill \null \\

\end{center}

\vfill
\hfill \thedate

\newpage

\thispagestyle{empty}

\section*{Remerciements}

\tableofcontents

\newpage

\pagenumbering{arabic}

\setlength{\parindent}{16pt}
\setlength{\parskip}{1ex}


\section{Avant-propos}

Le temps de téléscope étant précieux, nous nous sommes préparé, plusieurs mois avant le début du 
stage afin de profiter au mieux du temps d’observation qui nous était accordé.
Nous avons effectué un certain nombre de séances de préparation en amont. Il
s’agissait de prendre connaissance avec le sujet et les objets, les
observations souhaitées, le télescope et son fonctionnement…

TODO: Du blabla sur les sources, leur intérêt.

\begin{table}
  \caption[]{Liste des fréquences des principales transitions des molécules obervées}
  \label{frequence_transition}
  $$ 
  \begin{array}{ccc}
    \hline
    \hline
Molécule & Transition & Fréquence (GHz) \\
    \hline
H13CO+   &(1-0)    & 86.754285      \\
HCN      &(1-0)    & 88.631847      \\
HCO+     &(1-0)    & 89.188523      \\
N2H+     &(1-0)    & 93.176258      \\
CH3OH    &(2-1)    & 96.741377      \\
CS       &(2-1)	   & 97.980968      \\
C18O     &(1-0)    & 109.78218      \\
13CO     &(1-0)    & 110.20135      \\
12CO     &(1-0)    &  115.27120     \\
e-CH3OH  &	   & 216.945559     \\
SiO      &(5-4)    & 217.104984     \\
C18O     &(2-1)    & 219.560319     \\
H213CO   &(3-2)	   & 219.908525     \\
SO       &(5,6-4,5)& 219.949433     \\
13CO     &(2-1)    & 220.398686     \\
CH3CN    &12(5)-11(5)& 220.641096   \\
12CO     &(2-1)	   & 230.537990     \\
13CS     &(5-4)    & 231.22069      \\
N2D+     &(3-2)	   & 231.32166      \\
 \hline
  \end{array}
  $$
\end{table}
Côté télescope, le récepteur EMIR permet de faire cela. Il peut fonctionner
dans 4 bandes de fréquences différentes :

\begin{table}
  \caption[]{Caractéristiques d'EMIR}
  \label{freqency_EMIR}
  $$ 
  \begin{array}{cccc}
    \hline
    \hline
Bande & Fréquence (GHz) & LSB (GHz) & USB (GHz) \\
E0    &  73 - 117       & 73-97     & 89-117\\
E1    & 125 - 184       & 125-168   & 141-184\\
E2    & 202 - 274       & 202 (LO) - 268 (LI) & 217 (UI) - 274(UO)\\
E3    & 277 - 335       & 277 - 335 & 293-350\\
 \hline
  \end{array}
  $$
\end{table}


TODO: lister les récepteurs et leurs plages.

Pour chaque récepteur, il est possible de connecter plusieurs backends. La
plupart des backends sont indépendants entre eux (c’est a minima le cas de ceux
utilisés, FTS et VESPA), mais pour un détecteur donné, toutes les combinaisons
de branchements ne sont pas possibles. Il était donc d’autant plus important de
réfléchir à l’avance aux \textit{setups} que nous allions utiliser.

Après vérifications des différentes possibilités, et en tenant des éventuels
difficultés météo, nous avons préparés quatre ensembles de possibilités.

TODO: Insérer les figures avec des commentaires sur les choix.
%----------------------------------------------------------------- 
   \begin{figure}
   \centering
   \includegraphics[width=0.8\textwidth]{figures/specsetup-1mm.pdf}
      \caption{Plages de fréquence séléctionnées pour les observations à 1mm
              }
         \label{FigVibStab}
   \end{figure}
%-----------------------------------------------------------------

%----------------------------------------------------------------- 
   \begin{figure}
   \centering
   \includegraphics[width=0.8\textwidth]{figures/specsetup-3mm.pdf}
      \caption{Plages de fréquence séléctionnées pour les obersevations à 3mm
              }
         \label{FigVibStab}
   \end{figure}
%-----------------------------------------------------------------
Sur place, la découverte du mode \textit{parallel} de VESPA grâce à la présence
de Gabriel \textsc{Paubert} en tant qu’\textit{Astronomer on duty} nous a
permis d’optimiser plus encore nos \textit{setups}.

\section{Le voyage}

TODO: Une petite photo ? Ou deux. :p

QUESTION: Est-ce qu’on parle de la visite de l’intérieur du télescope, est-ce
qu’on le met ici ?

\section{Les observations}

TODO: W3OH, pointing, focus… Vérifier avec MIRA ce qu’on faisait. Xephem (ou un
nom du genre, je ne sais plus) pour vérifier la position du télescope.

%TODO: Quel temps faisait-il chaque jour, qu’est-ce qu’on a choisi comme setups
%par conséquent.
Le premier jour d'observation nous avons du utiliser le setup à 3 mm en raison des
conditions météorologiques peu propices à l'observation à plus peite longueur d'onde
(atmosphère trop opaque). Le second jour nous avons pu utiliser pleinement le setup à 1 mm.
Le troisième jour le temps étant mitigé, nous n'avons pas pris de risque et avons utilisé 
le setup mixte. Ceci nous à de plus permis d'utiliser au maximum les gammes de setup que
 nous avions préparés.

TODO: Expliquer les problèmes qu’on a eu (pas position de référence clean alors
qu’en fait elle le sont toutes a posteriori, erreur de position de référence
dans les observations d’une partie des données)

\section{Les données obtenues}

TODO: Lister les raies observées, leurs caractéristiques et par conséquent la
présence d’outflow. Parler de la raie tellurique.

TODO: Ajouter les figures de Jan.

\begin{figure}[h!]
\centering
\includegraphics[height=6cm]{figures/mapC17O.png}
\caption{Carte des flots observés de CO. En bleu le $^{12}CO$ ``blueshifté'' En rouge le flot ``redshifté''. Les isocontours de vitesse sont pris avec comme réference la vitesse de la raie principale de $^{12}CO$ (2->1) à 230.538 GHz, observée dans le setup à 1mm. Les bins de densité ont été réalisés avec la raie de $^{17}CO$, meilleure traceur du gaz au repos.}
\label{mapC17O}
\end{figure}

\section*{Conclusion}
\addcontentsline{toc}{section}{Conclusion}

TODO: C’était cool et intéressant. On ajoute une autre photo.

\end{document}
